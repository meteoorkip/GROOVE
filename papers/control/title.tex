\title{Controlled Rule Application using Failure Automata}
\author{Arend Rensink\inst{2} and Tom Staijen\inst{1}}
\authorrunning{A. Rensink and T. Staijen}

\institute{
Formal Methods and Tools Group, University of Twente, The Netherlands, \email{rensink@cs.utwente.nl}
\and 
Software Engineering Group, University of Twente, The Netherlands, 
\email{staijen@cs.utwente.nl}
}

\maketitle

\begin{abstract}
  Pure rule-based systems typically consist of a single, unstructured set of
  rules.  The behaviour of such systems is that all rules are applicable in
  every state. Rules can then only be forced into a certain order of
  application by adding special elements to the states, which are tested for
  within the rules. In other words, control over rule applicability is not
  explicit but has to be encoded in the state, which is bad for
  understandability and maintainability.

  In this paper, we study so-called \emph{control automata}, which can be added
  on top of pure rule systems. The resulting behaviour is defined as the
  product of the original state space and the control automaton.  Control
  automata include so-called failure transitions, representing the observation
  of the non-applicability of one or more rules.  The result is a reactive
  semantics for control expressions, which extends the usual input-output
  semantics in a consistent manner.

  Control automata may introduce artificial non-determinism into the behaviour,
  which is an undesirable effect. We introduce \emph{guarded control automata}
  to get rid of this effect, and we define a semantics-preserving
  transformation from ordinary control automata to guarded ones.
%   The contribution of this work consists of control automata with transitions
%   explicitly representing failures, that give a powerful means for controlling
%   rule-based systems. Also, a text-based, imperative control language --- with
%   some new control structures --- gives a simple yet powerful means for
%   specifying such automata.
\end{abstract}
