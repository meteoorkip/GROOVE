\subsection{Equivalence}
We will now show that guarded control automata serve their intended purpose, namely that the product of a system automaton with a control automaton is essentially the same as its product with the determinised guarded control automaton. "Essentially the same" means that they have the same language in terms of $(\Rule \times Id)$-traces. This is depicted in Figure \ref{fig:equiv} and Theorem \ref{th:equiv}.

\begin{figure}
\centering
\begin{tikzpicture}[shorten >=1pt,semithick,>=stealth']
%\tikzstyle{every state}=[shape=circle,draw]
\node		(s3)			{$\cA$};
\node		(s1)	[above=0.5cm of s3]		{$\times$};
\node		(s0)	[left=1.75cm of s1]		{$\cC$};
\node		(s2)	[right=1.4cm of s1]		{$\cA \times \cC$};
\node		(s5)	[below=0.5cm of s3]		{$\times$};
\node		(s4)	[left=1.1cm of s5]		{$\mathit{det}(\cC) = \cG$};
\node		(s6)	[right=1.4cm of s5] 		{$\cA \times \cG$};
\path[->]
(s0)		edge	node[above]	{}		(s1)
			edge	node[left]	{\scriptsize$\mathit{det}$}	(s4)
(s1)		edge	node[above]	{}		(s2)
(s4)		edge	node[below]	{}		(s5)
(s5)		edge	node[below]	{}		(s6)
(s3)		edge	node[left]	{}		(s1)
			edge	node[left]	{}		(s5)
(s6)		edge	node[right]	{\scriptsize \cL-equivalent}	(s2)
(s2)		edge	node			{}			(s6);
\end{tikzpicture}
\caption{Schematic Representation of the Equivalence Relationship}
\label{fig:equiv}
\end{figure}

\begin{theorem}\label{th:equiv}
For all system automata \cA~ and control automata \cC, $\cL(\cA \times \cC) = \cL(\cA \times \mathit{det}(\cC))$
\end{theorem}
%
To prove this, we show inclusions in both directions, using two distinct notions of simulation. Proof of the claimed properties is provided in Appendix \ref{app:proofs}. 

\begin{definition}[forward simulation]\label{def:rho}
Given two system automata $\cA_1, \cA_2$, a relationship $\rho \subseteq Q_1 \times Q_2$ is called a \emph{forward simulation} if:
\begin{align}\label{sim1}
&(q_{0_1}, q_{0_2}) \in \rho \: &
\end{align}
and for all: $(q_1,q_2) \in \rho$:
\begin{align}\label{sim2}
q_1 \Arrow{(n,i)} q_1' \: &\Rightarrow \:
\exists q_2 \Arrow{(n,i)} q_2' \wedge (q_1', q_2') \in \rho\\
\label{sim3}
\exists q_1 \Arrow{\epsilon} q_1' \in S_1 \: &\Rightarrow \:
q_2 \Arrow{\epsilon} q_2' \in S_2
\end{align}
\end{definition}

\begin{proposition}\label{prop:rhoeq1}
If there exists a forward simulation between $\cA_1$ and $\cA_2$, then $\Tr(\cA_1) \subseteq \Tr(\cA_2)$ and $\STr(\cA_1) \subseteq \STr(\cA_2)$.
\end{proposition}

\begin{proposition}\label{prop:rho}
Let $\cA$ be a system automaton and $\cC$ a control automaton; then the relation defined by $\rho = \setof{((q_\cA,q_\cC),(q_\cA,qs)) \mid q_\cC \in qs}$ is a forward simulation between $\cA \times \cC$ and $\cA \times \mathit{det}(\cC)$.
\end{proposition}

The other direction is also covered by a simulation.

\begin{definition}[reverse simulation]\label{def:rho'}
Given two system automata $\cA_1$ and $\cA_2$, $\rho \subseteq Q_2 \times 2^{Q_1}$ is called a \emph{reverse simulation} if:
\begin{align}\label{sim4}
&(q_{0,2}, \setof{q_{0,1}}) \in \rho &
\end{align}
and for all $(q_{\cA}, R) \in \rho$:
\begin{align}\label{sim5}
q_2 \arrow{(n,i)}_2 q_2' \: &\Rightarrow \: \exists (q_2', R') \in \rho.\forall r' \in R'. \exists r \in R.r \Arrow{(n,i)}_1 r'\\
\label{sim6}
q_2 \in S_2 \: &\Rightarrow \: \exists q_1 \in R. q_1 \Arrow{\epsilon} q_1' \in S_1
\end{align}
\end{definition}

\begin{proposition}\label{prop:rhoeq2}
If there exists a reverse simulation between $\cA_1$ and $\cA_2$, then $\Tr(\cA_2) \subseteq \Tr(\cA_1)$ and $\STr(\cA_2) \subseteq \STr(\cA_1)$.
\end{proposition}

\begin{proposition}\label{prop:rho'}
Let $\cA$ be a system automaton and $\cC$ a control automaton, and $\cG = det(\cC)$, then the relation defined by $\rho = \setof{ ((q_\cA,q_\cG),R) \mid \forall q_\cC \in q_\cG: (q_\cA,q_\cC) \in R}$ is a reverse simulation between $\cA \times \mathit{det}(\cC)$ and $\cA \times \cC$.
\end{proposition}
%
With the help of the above results, we can now prove the theorem.

\begin{proof}[Theorem \ref{th:equiv}]
From propositions \ref{prop:rhoeq1} and \ref{prop:rho} it follows that $\Tr(\cA \times \cC) \subseteq \Tr(\cA \times \mathit{det}(\cC))$ and $\STr(\cA \times \cC) \subseteq \STr(\cA \times \mathit{det}(\cC))$; from propositions \ref{prop:rhoeq2} and \ref{prop:rho'} it follows that the inverse inclusions also hold. This implies the proof obligation.
\end{proof}
