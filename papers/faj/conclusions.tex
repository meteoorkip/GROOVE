\section{Conclusion}\label{sec:gr-conclusion}

In this paper we have defined a graph transformation-based semantics for a simple object-oriented language with around advice. The specified language, Assignment Featherweight Java, leans itself very well for studying the implications of language extensions. We have extended this language with around advice bound to point-cuts that select certain instructions.\\
We have illustrated that a graph transformation based operational semantics is a formal specification technique and can be complete with respect to a certain reference semantics. Also, we have shown that it is better comprehensible then a specification using a mathematical notation, especially for those without a background in formal specification languages. This makes the approach suitable for the purpose of having a reference semantics.\\
We have demonstrated that a graph transformation based specification allows simulation of a program written in the language, if this program is represented as a graph as described in this paper. This gives a simple and intuitive view on the execution of the program, and opens the road towards applying existing verification methods such as analysis based on model checking.\\
We have shown that our simulation tool allows non-determi{\-}nistic execution, which opens possibilities for the verification of ordering conflicts between aspects.\\
Due to its executable nature, the graph transformation-based specification has led to the discovery of errors (e.g. a missing rule) in the specification used as a correctness criterion; we see these errors as an unfortunate consequence of a purely formal notation.\\
