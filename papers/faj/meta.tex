\begin{abstract}
In this paper we specify an operational semantics of Assignment Featherweight Java --- a minimal subset of Java with assignments --- with around advice, using graph transformations. We show that our specification is correct with respect to an existing semantics and that it has an intuitive visual representation; one can understand a graph-based semantics with less effort than the traditional notations of operational semantics.\\
A main advantage of graph transformations is their practical use: given a graph representation of an aspectual AFJ program, the behaviour can be (manually or automatically) simulated; the result of this simulation is a state space that can be used for analysis and verification, giving rise to an effective method for aspect program verification.\\
As a criterion for correctness, we use a structural operational semantics of this language from the so-called Common Aspect Semantics Base.
\end{abstract}

%\category{CR-number}{sub-category}{third-level}

%\terms
%term1, term2

\keywords
AOP, operational semantics, graph production system, graph transformation, simulation

%\newcommand{\todo}[1] {({\tt TODO:} {\emph #1})}
