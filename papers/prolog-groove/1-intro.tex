% -----------------------------------------------------------------------------
% 1-intro.tex
% -----------------------------------------------------------------------------

\section{Introduction}
\stlabel{intro}

The practical value of graph transformation (GT) is especially determined by the
fact that graphs are a very general, widely applicable mathematical structure.
Virtually every artefact can be understood in terms of entities and relations
between them, which makes it a graph; and consequently, changes in such an
artefact can be specified through GT rules.

On the other hand, capability does not automatically imply suitability. For
instance, though it is possible to express structural properties as (nested)
graph conditions -- see, for instance, \cite{Rensink2004,HabelPenRen2006} -- in
practice, if one wants to query a given structure, writing graphical conditions
to express and test for such queries is not always the most obvious or effective
way to go about it. This is particularly true if the queries have not been
predefined but are user-provided.  Instead, there are dedicated languages
suitable for querying relational structures, such as, for instance, \SQL or
\PROLOG.

The need for a powerful and flexible query language becomes even more clear
when one wants to combine static (structural) properties with dynamic ones, so
as to include the future or past evolution of the structure. For instance,
temporal logic has been especially introduced to express dynamic properties and
check them efficiently (see \cite{BaierKat2008} for an overview). However,
besides lacking accessibility, temporal logic is \emph{propositional}, meaning
that it takes structural properties as basic building blocks; there is very
little work on logics that can freely mix static and dynamic aspects of a
system.

\medskip\noindent An example domain that requires this combination of static
and dynamic aspects is \emph{feature modelling}. A feature model is a graph in which
nodes represent possible features (of some system under design) and edges express
that one feature requires another, is in conflict, or is related in some
other way. Graph transformation can be used to actually select features (in
such a way that the constraints are met). The outcome is a (partially) resolved
model, the quality of which is not only determined by the choices actually made
but also by the possible choices still remaining. Thus, one would like to query
a feature model for both its static properties (the choices actually made)
and for its dynamic properties (the potential further transformation steps).
 
\medskip\noindent In this paper, we describe how one can use \PROLOG to query
static and dynamic properties of graphs, simultaneously and uniformly.  Besides
the transformed graphs this requires a graph transition system (GTS), which is
itself a graph with nodes corresponding to state graphs and edges to rule
applications. The basic building block of \PROLOG is a \emph{predicate}, which
expresses a relation between its arguments. Example predicates in our setting
are:
\begin{itemize}\noitemsep
\item The relation between a graph and its nodes or edges;
\item The relation between an edge and its source or target node, or its label;
\item The relation between a state of the GTS and its corresponding graph;
\item The relation between one state of the GTS and the next.
\end{itemize}

\medskip\noindent A collection of \Prolog predicates forms a
\emph{knowledge-base}, which is queried during the analysis of a GTS. Using an
extension of the transformation tool \GROOVE that supports \PROLOG queries, we
demonstrate the capabilities of this approach on a case study based on feature
modelling. This domain was chosen due to its applicability on the development
process of software industries.

\medskip\noindent The paper is organised as follows. We first present the basic
concepts for querying graphs using \PROLOG (\stref{prolog}); then we describe
the application to feature modelling in \stref{problem}. An analysis of the
results can be found in \stref{discussion}. Conclusion and ideas for future work
are given in \stref{conclusion}.

%
%%% Local Variables:
%%% mode: latex
%%% TeX-master: "main"
%%% End:
%