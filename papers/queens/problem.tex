% -----------------------------------------------------------------------------
% problem.tex
% -----------------------------------------------------------------------------

\section{Problem}
\stlabel{problem}

The $N$-queens puzzle requires the placement of $N$ queens on an $N \times N$
chessboard in such a way that no queen can attack any other. In chess, a queen
is able to move any number of squares in any of the eight directions, and thus
a solution to the puzzle requires that no two queens share the same row,
column, or diagonal.  In the usual instance of the problem a standard sized
chessboard is used, and therefore we have $N = 8$. \fref{ex-sol-8} shows one
solution for this instance.

\putfig[1.25]{ex-sol-8}{Solution for the 8-queens puzzle.}

The 8-queens puzzle has 92 \emph{distinct} solutions. If we collapse (count as
one) solutions that differ only by rotations and reflections of the board, the
puzzle has 12 \emph{unique} solutions. Given that there are $\binom{64}{8} =
4,426,165,368$ possible arrangements of eight queens on a $8 \times 8$ board,
it is unfeasible to use a simple brute force algorithm that generates a
configuration and tests if it is a solution.

Finding all (unique) solutions to the 8-queens puzzle is a classic example of a
simple but nontrivial problem, which has been used many times as an exercise in
algorithm design. In 1972, Dijkstra \cite{Dij72} presented a solution using a
recursive depth-first backtracking algorithm. Since then many other programming
techniques such as constraint and logic programming have used this problem to
illustrate their features.

The number of distinct and unique solutions for $0 < N < 27$ are respectively
given in sequences A000170\footnote{\url{http://oeis.org/A000170}} and
A002562\footnote{\url{http://oeis.org/A002562}} of the On-Line Encyclopedia of
Integer Sequences (OEIS). This information is summarised in \tabref{sol-count}
for $N$ up to 10. A variant of the problem consists of finding the minimum
number of $k$ queens needed to occupy or attack all squares of a $N \times N$
board. For the 8-queens puzzle the answer is $k = 5$, with 91  unique
configurations. The summary for this problem variant is given in the last two
lines of \tabref{sol-count}.

\begin{table}[t]
\centering
\begin{tabular}{@{}l@{\enspace}l|rrrrrrrrrr}
\hline
\multicolumn{2}{r|}{$N$}
 & \bf 1 & \bf 2 & \bf 3 & \bf 4 & \bf 5 & \bf 6 &
\bf 7 & \bf 8 & \bf 9 & \bf 10\\
\hline
\multirow{2}{*}{\bf $N$-queens problem}
 & \bf distinct solutions & 1 & 0 & 0 & 2 & 10 & 4 & 40 & 92 & 352 & 724\\
 & \bf unique solutions & 1 & 0 & 0 & 1 & 2 & 1 & 6 & 12 & 46 & 92\\
\hline
\multirow{2}{*}{\bf $k$-queens variant}
 & \bf minimum $k$ & 1 & 1 & 1 & 3 & 3 & 4 & 4 & 5 & 5 & 5\\
 & \bf unique solutions & 1 & 1 & 1 & 2 & 2 & 17 & 1 & 91 & 16 & 1\\
\hline
\end{tabular}
\caption{Summary of (known) solution counts for the different problem
instances.}
\tablabel{sol-count}
\end{table}
