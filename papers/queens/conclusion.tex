% -----------------------------------------------------------------------------
% conclusion.tex
% -----------------------------------------------------------------------------

\section{Conclusion}
\stlabel{conclusion}

We usually advertise \GROOVE as a flexible modelling tool, specially suited for
rapid prototyping~\cite{GMR+12}. Indeed, at least for us, \GROOVE is flexible
and easy to use, but our own tool experience is obviously not a good
representative of the ``average'' user. The goal of this paper was to bridge
the gap (or at least shrink it) between novice and more advanced \GROOVE users
by pointing to information that we, as tool developers, consider important but
that up to now was not properly available and documented.

As a side note, we are aware that this text contains a high number of imprecise
terms such as ``usually'', ``generally'', ``in some cases'', etc. We hope for
some understanding from the readers in this matter, as this language use is
an unfortunate side-effect of writing about best practices and guidelines. For
every rule there is an exception, and making clear-cut statements about certain
topics would make them dubious at best, and plain wrong at worst.

The $N$-queens puzzle is a good representative for the class of combinatorial
problems that can be tackled using \GROOVE, and therefore the information
presented here is geared towards this kind of problem. Future work w.r.t. this
form of documentation points to the development of similar tool usage
guidelines for other problem classes, such as model transformation. A more
extensive line of future work points towards a tool usability study, where an
empirical evaluation with novice GROOVE users could help identify the strengths
and weaknesses of the tool.

\paragraph{Related tools.} In this paper we have concentrated on the
capabilities and features of \GROOVE; as stated in the beginning of
\stref{discussion}, our guidelines have been formulated especially with this in
mind. Of the rich set of other graph transformation-based tools, we believe that
\HENSHIN~\cite{HENSHIN} comes closest in matching the particular capabilities of
\GROOVE. For a very comprehensive overview of other modelling graph-based tools
see~\cite{Jakumeit+2013}.

\paragraph{Availability.} The experiments presented in this paper were
performed with \GROOVE version 4.9.2, available at
\url{http://groove.cs.utwente.nl}. The grammar for solving the $N$ queens puzzle
can also be downloaded at the same address.
