\section{Definitions and Conversions}

In this section we define the two relevant types of graphs: edge-labelled
graphs as used in \GROOVE, and node-coloured graphs as used in \NAUTY and
\BLISS, and we define two isomorphism-preserving conversions between these
classes. All graphs are directed.

We assume a totally ordered universe of labels $\Lab$. These are used both for
edge labels and for node colours. For a subset $L\subseteq \Lab$, we use
$\succ_L$ to denote the (partial) successor function mapping each $l\in L$ to
its next higher label in $L$ (with respect to the total ordering over $\Lab$).

First we formalise edge-labelled graphs and the corresponding notion of
isomorphism. Isomorphism of edge-labelled graphs is a node bijection that
preserves and reflects edges (including their labels).

\begin{definition}[Edge-labelled graph]
  An \emph{edge-labelled graph} $G$ is a tuple $\langle V_G, E_G
  \rangle$ with a finite nonempty set of nodes $V_G$ and a finite set of edges
  $E_G \subseteq V_G \times \Lab \times V_G$.  The class of edge-labelled
  graphs is denoted $\GL$.
\end{definition}

\begin{definition}[Isomorphism of edge-labelled graphs]
  Let $G, H \in \GL$. A bijective function $f: V_G \rightarrow V_H$ is called
  an \emph{isomorphism} if $(v_1,l,v_2) \in E_G$ if and only if
  $(f(v_1),l,f(v_2)) \in E_H$. If such a function exists, $G$ and $H$ are
  called \emph{isomorphic}, denoted $G \cong H$.
\end{definition}
%
In node-coloured graphs, on the other hand, each node has a colour, and
isomorphism also preserves and reflects node colours.

\begin{definition}[Node-coloured graph]
  A \emph{node-coloured graph} $G$ is a tuple $\langle V_G, E_G, c_G
  \rangle$ with a finite nonempty set of nodes $V_G$, a set of edges $E_G
  \subseteq V_G \times V_G$, and a colour function $c_G : V_G \rightarrow \Lab$.
  The class of node-coloured graphs is denoted $\GC$.
\end{definition}

\begin{definition}[Isomorphism of node-coloured graphs]
  Let $G, H \in \GC$. A bijective function $f: V_G \rightarrow V_H$ is called
  an \emph{isomorphism} if for all $v \in V_G : c_G(v) = c_H(f(v))$, and
  $(v_1,v_2) \in E_G$ if and only if $(f(v_1),f(v_2)) \in E_H$. If such a
  function exists, $G$ and $H$ are called \emph{isomorphic}, denoted $G \cong
  H$.
\end{definition}

% I don't think this is really necessary... EZ
% \NAUTY and \BLISS are based on the computation of canonical forms of graphs.
% By definition, a set of isomorphic graphs has only one canonical form, which
% serves as an unique representative of the set.
%
% \begin{definition}[Canonical representation and canonical form]
% For a class of graphs $\G$, a \emph{canonical representation} function $\can :
% \G \rightarrow \G$ computes an isomorphism invariant graph representative for
% each graph, such that for every pair of graphs $G,H \in \G$, $\can(G) =
% \can(H)$ if and only if $G \cong H$. If $\can$ is a canonical representation
% function, $\can(G)$ is called the \emph{canonical form} of $G$.
% \end{definition}
