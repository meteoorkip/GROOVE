% EZ: Uncomment this to draw margins on the document.
% \AddToShipoutPicture{\AtTextLowerLeft{%
% \framebox(\LenToUnit{\textwidth},\LenToUnit{\textheight}){}}%
% }

\mainmatter  % start of an individual contribution

% first the title is needed
\title{Isomorphism Checking Methods for Directed Edge-Labelled Graphs}

% a short form should be given in case it is too long for the running head
\titlerunning{Isomorphism Checking for Directed Edge-Labelled Graphs}

\author{Gijs Kant \and Arend Rensink \and Eduardo Zambon}

\authorrunning{G. Kant  \and A. Rensink \and E. Zambon}
% (feature abused for this document to repeat the title also on left hand pages)

% the affiliations are given next; don't give your e-mail address
% unless you accept that it will be published
\institute{Department of Computer Science, University of Twente \\
P.O. Box 217, 7500 AE Enschede, The Netherlands \\
\email{\{kant, rensink, zambon\}@cs.utwente.nl}}

% AR: Commented out, the title is different and do we really need this?
\toctitle{Isomorphism Checking Methods for Directed Edge-Labelled Graphs}
\tocauthor{G. Kant, A. Rensink, E. Zambon}
\maketitle

\begin{abstract}
  For the purpose of state space reduction, we want to efficiently find an
  isomorphic representative of a given graph in a set of graphs. The graphs are
  directed and edge-labelled. In this paper we compare our own direct
  implementation of isomorphism checking with a solution that uses existing,
  state-of-the-art tools.
\end{abstract}

% \keywords{Graph Isomorphism, Edge-Labelled Graphs, Canonical Graphs}
