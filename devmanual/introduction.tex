\chapter{Introduction}
\chlabel{Introduction}

Developing and maintaining tools can be very hard. Especially when the group of people doing this work is changing frequently. In an academic setting this is mostly the case. Moreover, in academia, tools are often extended by MSc students writing some specific component for their particular project. In such projects most effort is spend on getting the tool do the job (pretty) correctly while not so much time is reserved for making the implementation well-structured, maintainable, and resource efficient.

The main reason for implementations being not so well-structured is because it is very difficult for a student to get a quick overview of a rather (theoretically or conceptually) complex tool. This document is a first attempt to tackling this problem by providing new members of the development team from some pointers to an introduction on the theoretical background, basic insight in the structural part of the tool, and some coding guidelines each member should apply stricly.

\section{Overview}

The remaining of this document is structured as follows. \chref{Background} describes some background about the original project this tool is developed in. \chref{The GROOVE Tool Set} gives an overview of the different components available within the \GROOVE Tool Set and describes their relationships. Thereafter, \chref{Tool Architecture} gives a structural overview of the internals of the tool. % In \chref{License} we will shortly discuss under which license the \GROOVE Tool Set is distributed and why.
Finally, \chref{Guidelines} enumerates some coding guidelines that should be applied by each developer after which we conclude with some small remarks.
